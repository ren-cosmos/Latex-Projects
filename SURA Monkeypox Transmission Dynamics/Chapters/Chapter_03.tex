%======================================================================
\chapter{Analysis}
%======================================================================

%----------------------------------------------------------------------
\section{Positivity}
%----------------------------------------------------------------------
In equations \ref{eqn:exposed} and \ref{eqn:vaccinated}, we can clearly see that the integrand is positive, and thus, the integral i.e $E_{h}(t)$ and $V_{h}(t)$ will also be positive.

Now, assume that $S_{h}(0)>0$. From equation \ref{meq1}, we get
\begin{align}
S_{h}'(t) = -kS_{h}(t)+a_{h}-\int_{0}^{t} bS_{h}(\mu)e^{-d_{h}(t-\mu)}P'(t-\mu) \,d\mu \label{positive}
\end{align}
where $ k = d_{h}+b+\beta_{h}I_{r}+\alpha_{h}I_{h} $
In equation \ref{positive}, we see that $a_{h}-\int_{0}^{t} bS_{h}(\mu)e^{-d_{h}(t-\mu)}P'(t-\mu) \,d\mu$ is positive. Hence, we get
\[S_{h}'(t)\geq-kS_{h}(t)\]
By using Gronwall Inequality, we get
\[S_{h}(t) \geq S_{h}(0)e^{-\int_{}^{} k \,dt}\]
\[\Rightarrow S_{h}(t) \geq 0 \hspace{10mm} \forall \hspace{1mm} t \geq 0\]

Similarly by assuming that $I_{h}(0) \geq 0$, $R_{h}(0) \geq 0$, $S_{r}(0) \geq 0$, $I_{r}(0) \geq 0$ and $R_{r}(0) \geq 0$, we can show that $I_{h}$, $R_{h}$, $S_{r}$, $I_{r}$ and $R_{r}$ are all positive respectively $\forall \hspace{1mm} t \geq 0$.

%----------------------------------------------------------------------
\section{Boundedness}
%----------------------------------------------------------------------
Let $R = \lbrace(S_{h}(t), V_{h}(t), E_{h}(t), I_{h}(t), R_{h}(t), S_{r}(t), I_{r}(t), R_{r}(t)) \in \Re^8+\rbrace$ represent the region. By adding the equations \ref{meq1} - \ref{meq5}, we get
\[N_{h}'(t) = a_{h}-d_{h}N_{h}(t)-\delta_{h}I_{h}(t)\]
\[\Rightarrow N_{h}'(t) \leq a_{h}-d_{h}N_{h}(t)\]
By using Gronwall Inequality, we get
\[N_{h}(t) \leq \frac{a_{h}}{d_{h}} + (N_{h}(0)-\frac{a_{h}}{d_{h}})e^{-d_{h}t}\]
now if $N_{h}(0) \leq \frac{a_{h}}{d_{h}}$, then
\[N_{h}(t) \leq \frac{a_{h}}{d_{h}}\]
and if $N_{h}(0) \geq \frac{a_{h}}{d_{h}}$, then
\[N_{h}(t) \leq N_{h}(0)\]
and similarly, we can show that\\
if $N_{r}(0) \leq \frac{a_{r}}{d_{r}}$, then
\[N_{r}(t) \leq \frac{a_{r}}{d_{r}}\]
and if $N_{h}(0) \geq \frac{a_{r}}{d_{r}}$, then
\[N_{r}(t) \leq N_{r}(0)\]
Now since both $N_{h}(t)$ and $N_{r}(t)$ are bounded, all the solutions are also bounded.

%----------------------------------------------------------------------
\section{Existence of DFE and EE}
%----------------------------------------------------------------------
Disease free equilibrium exists at $(S_{h}(t), V_{h}(t), 0, 0, 0, S_{r}(t), 0, 0)$. Equation \ref{meq6} at the disease free equilibrium point becomes
\[a_{r}-d_{r}S_{r}(t) = 0\]
\[\Rightarrow S_{r}(t) = \frac{a_{r}}{d_{r}}\]
Now, consider equation \ref{meq1}. At the disease free equilibrium point, equation \ref{meq1} becomes
\[a_{h}-d_{h}S_{h}(t)-bS_{h}(t)-\int_{0}^{t} bS_{h}(\mu)e^{-d_{h}(t-\mu)}P'(t-\mu) \,d\mu = 0\]
At $t \rightarrow \infty$, by assuming that $S_{h}(t) \rightarrow S^{*}_{h}(t)$, we get
\[a_{h}-d_{h}S^{*}_{h}(t)-bS^{*}_{h}(t)-bS^{*}_{h}(t)\int_{0}^{t}e^{-d_{h}(t-\mu)}P'(t-\mu) \,d\mu = 0\]
%-------------------------------------------------------------
By substituting $v = t-\mu$, we get
\[a_{h}-d_{h}S^{*}_{h}(t)-bS^{*}_{h}(t)-bS^{*}_{h}(t)\int_{0}^{t}e^{-d_{h}(v)}P'(v) \,d\mu = 0\]
%-------------------------------------------------------------
\[\Rightarrow a_{h}-d_{h}S^{*}_{h}(t)-bS^{*}_{h}(t)\left(\left[e^{-d_{h}v}P(v)\right]^{t}_{0}-\int_{0}^{t}(-d_{h})P(v)e^{-d_{h}v} \,dv\right) = 0\]
%-------------------------------------------------------------
\[\Rightarrow d_{h}(1-bP^{*})S^{*}_{h}(t) = a_{h}\]
where $P^{*} = -\int_{0}^{\infty}e^{-d_{h}v} \,dP(v) $ is the average time an individual spends in the vaccinated class
%-------------------------------------------------------------
\[\Rightarrow S^{*}_{h}(t) = \frac{a_{h}}{d_{h}(1-bP^{*})}\]
%-------------------------------------------------------------
Similarly, we can find that
\[V^{*}_{h}(t) = \frac{-ba_{h}P^{*}}{d_{h}(1-bP^{*})}\]

Hence, disease free equilibrium exists at $\left(\frac{a_{h}}{d_{h}(1-bP^{*})},\frac{-ba_{h}P^{*}}{d_{h}(1-bP^{*})},0,0,0,\frac{a_{r}}{d_{r}},0,0\right)$

%----------------------------------------------------------------------
\section{Stability Analysis}
%----------------------------------------------------------------------
Here, you need to check the type of equilibrium point and identify whether it is stable, unstable or semi-stable.