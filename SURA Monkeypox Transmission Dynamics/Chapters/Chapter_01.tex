%======================================================================
\chapter{Introduction}
%======================================================================
Monkeypox is a viral zoonosis (a virus transmitted to humans from animals) with symptoms very similar to those seen in the past in smallpox patients, although it is clinically less severe. It is caused by the monkeypox virus which belongs to the Orthopoxvirus genus of the Poxviridae family. The name monkeypox originates from the initial discovery of the virus in monkeys in Statens Serum Institute, Copenhagen Denmark, in 1958. The first human case was identified in a young child in the Democratic Republic of the Congo in 1970.

%----------------------------------------------------------------------
\section{Background}
%----------------------------------------------------------------------
Monkeypox is commonly found in central and west Africa where there are tropical rainforests and where animals that may carry the virus typically live. People with monkeypox are occasionally identified in other countries outside of central and west Africa, following travel from regions where monkeypox is endemic.
Monkeypox virus is transmitted from one person to another by close contact with lesions, body fluids, respiratory droplets and contaminated materials such as bedding. The incubation period of monkeypox is usually from 6 to 13 days but can range from 5 to 21 days.
Various animal species have been identified as susceptible to the monkeypox virus. Uncertainty remains on the natural history of the monkeypox virus and further studies are needed to identify the reservoir(s) and how virus circulation is maintained in nature. Eating inadequately cooked meat and other animal products of infected animals is a possible risk factor.
Monkeypox is usually self-limiting but there is likely to be little immunity to monkeypox among people living in non-endemic countries since the virus has not previously been identified in those populations. There are two variants of the monkeypox virus: the West African variant and the Congo Basin (Central African) variant. The Congo Basin variant appears to cause severe disease more frequently with a case fatality ratio (CFR) previously reported of up to around 10\%. Currently, the Democratic Republic of the Congo is reporting a CFR among suspected cases of around 3\%. The West African clade has in the past been associated with an overall lower CFR of around 1\% in a generally younger population in the African setting. Since 2017, the few deaths of persons with monkeypox in West Africa have been associated with young age or an untreated HIV infection.

%----------------------------------------------------------------------
\section{Symptoms}
%----------------------------------------------------------------------
Symptoms of monkeypox typically include a fever, intense headache, muscle aches, back pain, low energy, swollen lymph nodes, and a skin rash or lesions. The rash usually begins within one to three days of the start of a fever. Lesions can be flat or slightly raised, filled with clear or yellowish fluid, and can then crust, dry up and fall off. The number of lesions on one person can range from a few to several thousand. Symptoms typically last between 2 to 4 weeks and go away on their own without treatment.

%----------------------------------------------------------------------
\section{OutBreak in Endemic and Non-Endemic Countries}
%----------------------------------------------------------------------
Since 13 May 2022, monkeypox has been appearing in 23 Member States that are not endemic to the monkeypox virus, across four WHO regions. Epidemiological investigations are ongoing. The vast majority of reported cases so far have no established travel links to an endemic area and have presented through primary care or sexual health services. The identification of confirmed and suspected cases of monkeypox with no direct travel links to an endemic area is atypical. Early epidemiology of initial cases notified to WHO by countries shows that cases have been mainly reported amongst men who have sex with men (MSM). One case of monkeypox in a non-endemic country is considered an outbreak. The sudden appearance of monkeypox simultaneously in several non-endemic countries suggests that there may have been undetected transmission for some time as well as recent amplifying events.
As of 26 May, a cumulative total of 257 laboratory-confirmed cases and around 120 suspected cases have been reported to WHO. No deaths have been reported.
The situation is evolving rapidly and WHO expects that there will be more cases identified as surveillance expands in non-endemic countries, as well as in countries known to be endemic who have not recently been reporting cases.

%----------------------------------------------------------------------
\section{Vaccinations}
%----------------------------------------------------------------------
There are several vaccines available for prevention of smallpox that also provide some protection against monkeypox. A newer vaccine that was developed for smallpox (MVA-BN, also known as Imvamune, Imvanex or Jynneos) was approved in 2019 for use in preventing monkeypox and is not yet widely available. Vaccinia is also known to deliver long-lasting immunity against
monkeypox, with 85\% efficacy.
