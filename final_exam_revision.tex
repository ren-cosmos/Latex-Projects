\documentclass[12pt]{article}
\usepackage[margin=0in]{geometry}
\usepackage{amsmath}
\usepackage{amssymb}
\title{Winter 2022 - Sample Final Exam}
\begin{document}
\maketitle
	\paragraph{1.}
	Evaluate the following limits:
	\\
	\\
	\textbf{a)} $ \displaystyle \lim_{n \to \infty} \frac{(\sqrt{n^2+1}+n)^2}{\sqrt[3]{(n^6+1)}}$\\
	\\
	\textbf{b)} $ \displaystyle \lim_{n \to \infty} \frac{(n+2)! + (n+1)!}{(n+2)! - (n+1)!}$\\
	\\
	\textbf{c)} $ \displaystyle \lim_{x \to 1} \frac{x+2}{x^2-5x+4} + \frac{x-4}{3(x^2-3x+2)}$\\
	\\
	\textbf{d)} $ \displaystyle \lim_{\alpha \to \beta} \frac{\sin^2(\alpha)-\sin^2(\beta)}{\alpha^2 - \beta^2}$\\
	\\
	\textbf{e)} $ \displaystyle \lim_{x \to 0} \frac{1-\cos(x)\sqrt{\cos(2x)}}{x^2}$\\
	
	\paragraph{2.}
	Check whether the function
	\textit{
	$$f(x) = - \hspace{4pt}\frac{2^\frac{1}{x}-1}{2^\frac{1}{x}+1}$$
	}
	\hspace{12pt} is continuous at $x=0$. If not, then classify the type of discontinuity.
	
	\paragraph{3.}
	Using the definition of derivatives, find the derivative of
	$$f(x) =  \frac{1}{\sqrt{2x+1}} $$
	\hspace{16pt} and also find the equations of tangent and normal at $x=0$.
	
	\paragraph{4.}
	Find the derivatives of the following:
	\\
	\\
	\textbf{a)} $ \displaystyle f(x)=x^2e^{\sqrt{x}}$\\
	\\
	\textbf{b)} $ \displaystyle f(x) = \cos^2{x^2}+\sinh^2{x^2}$\\
	\\
	\textbf{c)} $ \displaystyle y+\sin(y)=\cos(\frac{\pi}{2}+x)$ at $x=0$.\\
	\\
	\textbf{d)} $ \displaystyle f(x)=\sin(\cosh(\sec^2(x^2+x+1)))$\\
	\\
	\textbf{e)} $ \displaystyle f(x)=\frac{e^{x^2}}{x+1}$\\
	
	\paragraph{5.}
	An airplane flying at an altitude of $12$ km passed directly over a radar antenna. When
	the distance from the radar antenna to the plane was $13$ km, the radar detected that the
	distance to the airplane was changing at a rate of $20$ km per hour. What was the speed of
	the plane?
	
	\paragraph{6.}
	Consider the following function
		$\displaystyle f(x) = \frac{5x(x+4)}{(x+1)^2}$ with derivatives
		\begin{center}
		$\displaystyle f'(x) = \frac{40}{(x+2)^3}$ \hspace{16pt} and \hspace{16pt} $\displaystyle f"(x) = \frac{-120}{(x+2)^4}$\\
		\end{center}
		
	\textbf{a)} find the points of discontinuity, including the  vertical asymptotes\\
	\\
	\textbf{b)} find the horizontal asymptotes of$ \displaystyle f(x).$\\
	\\
	\textbf{c)} Find the $x$-intercept and $y$-intercept of $f(x)$.\\
	\\
	\textbf{d)} Determine the interval on which $f(x)$ is increasing or decreasing, and classify any relative (local) extrema.\\
	\\
	\textbf{e)} Determine the interval on which $f(x)$ is concave upward or concave downward, and find any point of inflexion.\\
	\\
	\textbf{f)} Sketch the graph of $f(x)$. Label your graph carefully.
	
	\paragraph{7.}
	A square piece of tin of side $24cm$ is to be made into a box without top by cutting a square from each corner and folding up the flaps to form a box. What should be the side of the square to be cut off so that the volume of the box is maximum? Also, find this maximum volume.

	\paragraph{8.}
	Use L'Hopitals rule to evaluate the following:
	$$ \lim_{x\to\infty} f(x) \frac{e^{x^2}-x+1}{x^2-x+sin(x^2)}$$
	
	\paragraph{9.}
	Use the definition of derivatives to prove that:
	$$[f(x)+g(x)]' = f'(x)+g'(x)$$
	
	\paragraph{10.}
	Give a piecewise function that is defined everywhere, except at $x=0$ and $x=-6$ which are also the points where it's behaviour changes. The one sided limits from the left and the right should exist as $x$ approaches both $0$ and $-6$, but the limit should exist only as $x$ approaches $-6$.
\end{document}

