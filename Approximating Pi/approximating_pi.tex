\documentclass[12pt]{article}
\usepackage{2130}
\usepackage{amsmath}

\lhead{Project 2}
\rhead{MATH 2130} % appears in the header on the right
\lfoot{Shivam Grover} % appears in the footer on the left
\rfoot{\thepage} % page number, in the footer on the right
\underheadoverfoot % dividing lines: under the header and below the footer

\begin{document}
\begin{titlepage}
\begin{center}
\large APPROXIMATING $\displaystyle \pi$
\end{center}
\vspace{6cm}
\hfill\begin{tabular}{ll}
MATH 2130 & Project 2 \\
Submitted by: & Shivam Grover \\
& \#202045704 \\
Submitted to: & Dr. Shannon Patrick Sullivan \\
Mar. 17, 2022

\end{tabular}
\end{titlepage}

\bf{ABSTRACT} \\
Sentence 1: blah blah blah is the topic of this study.\\
Sentence 2: dash is the approach or the method we are going to use to study this topic.\\
Sentence 3: blah blah blah are the results that we obtain.\\
Sentence 4: blah blah blah are the major conclusions that we draw.\\

\section {INTRODUCTION}
$\pi$ is an irrational number which is defined as the ratio of the circumference and the diameter of a circle. Surprisingly, it is independent of the size of a circle and it was around 2000 BCE when the Babylonians and the Egyptians realized that $\pi$ is a constant. Babylonians used $\frac{25}{8}$ as an approximate of $\pi$, whereas the Egyptians estimated it to be $4\cdot(\frac{8}{9})^2$. Later, a more accurate value $\frac{22}{7}$ came into everyday use. Over a period of time, significant efforts were made to improve the estimation of $\pi$ and nowadays, computing millions of digits of $\pi$  is considered as one of the most popular methods of testing the latest computer architectures. But before the realm of modern computing, many mathematicians like Archimedes had already formed algorithms for computing $\pi$ with controlled accuracy. But, what do we mean by controlled accuracy?


\section {TECHNICAL DETAILS}
It gives the mathematical background of the topic and states the methodology used for studying the topic. Can also include computer program. If run for 5 or 6 pages, then you should mke sub sections for background, methodology and code. You should explain the inputs of the program, explain what the programs does on higher level and what are the outputs of the program.

\section {RESULTS}
we can represent the results in the form of a table, or graph, etc.

Graph (approx value vs number of sides) representing the result of each iteration would be a better way to represent the results in this project.


\section {ANALYSIS}
answer any questions which were initially posed, give the reader a deeper understanding of the topic.


\section {CONCLUSION}
summarize findings, acknowledge potential future research, indicate any results which could not be obtained.


\section {REFERENCES}
should not be numbered

\section {APPENDICES}
includes computer code, any other related info which is not strongly related but you want the reader to have access to. Number them in a different notation from sections.


\end{document}

