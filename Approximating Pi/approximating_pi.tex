\documentclass[12pt]{article}
\usepackage{2130}
\usepackage{amsmath}
\usepackage{tikz}
\usepackage{graphicx}

\lhead{Project 2}
\rhead{MATH 2130} % appears in the header on the right
\lfoot{Shivam Grover} % appears in the footer on the left
\rfoot{\thepage} % page number, in the footer on the right
\underheadoverfoot % dividing lines: under the header and below the footer

\begin{document}
\begin{titlepage}
\begin{center}
\large Approximating $\displaystyle \pi$
\end{center}
\vspace{6cm}
\hfill\begin{tabular}{ll}
MATH 2130 & Project 2 \\
Submitted by: & Shivam Grover \\
& \#202045704 \\
Submitted to: & Dr. Shannon Patrick Sullivan \\
Mar. 17, 2022

\end{tabular}
\end{titlepage}

\bf{ABSTRACT} \\
We aim to demonstrate in this paper how Archimedes devices a simple and clear
algorithm of approximating the value of $\pi$.\\
Sentence 2: dash is the approach or the method we are going to use to study this topic.\\
Sentence 3: blah blah blah are the results that we obtain.\\
Sentence 4: blah blah blah are the major conclusions that we draw.\\

\section {Introduction}
$\pi$ (also known as Archimedes constant) is an irrational number which is defined as the ratio of the circumference and the diameter of a circle. Surprisingly, it is independent of the size of a circle and it was around 2000 BCE when the Babylonians and the Egyptians realized that $\pi$ is a constant. Babylonians used $\frac{25}{8}$ as an approximate of $\pi$, whereas the Egyptians estimated it to be $4\cdot(\frac{8}{9})^2$. Later, a more accurate value $\frac{22}{7}$ came into everyday use. Over a period of time, significant efforts were made to improve the estimation of $\pi$ and nowadays, computing millions of digits of $\pi$  is considered as one of the most popular methods of testing the latest computer architectures. But before the realm of modern computing, many mathematicians like Archimedes had already formed algorithms for computing $\pi$.


\section {Technical Details}
\iffalse It gives the mathematical background of the topic and states the methodology used for studying the topic. Can also include computer program. If run for 5 or 6 pages, then you should make sub sections for background, methodology and code. You should explain the inputs of the program, explain what the programs does on higher level and what are the outputs of the program.\fi

\subsection{Background}
Archimedes developed a simple but ingenious algorithm to approximate $\pi$. He found that we can approximate $\pi$ by using a n-sided regular polygon (also called n-gon) for a given circle. The lower bound of $\pi$ can be approximated by using an inscribed regular n-gon inside the circle. Whereas, the upper bound of $\pi$ can be estimated by using a circumscribed regular n-gon. He realised that the perimeter of the regular n-gon can be used to estimate the circumference of the given circle and hence $\pi$ can be approximated by taking the ratio of the perimeter of the regular n-gon and the diameter of the given circle.\\

% show a figure of inscribed and circumscribed regular hexagon
\begin{figure}[h!]
  \includegraphics[width=\linewidth]{inscribed_and_circumscribed_hexagons.pdf}
  \caption{Inscribed and circumscribed hexagon}
  \label{fig:ics}
\end{figure}
Figure \ref{fig:ics} shows inscribed and circumscribed regular hexagons where it can be seen that circumference of a circle lies between the perimeters of inscribed and circumscribed hexagons.\\
\bigskip

\subsection{Methodology}

% Show the figure 2 containing an inscribed hexagon in a unit circle.
\begin{figure}[h!]
  \includegraphics[width=\linewidth]{hexagon_inscribed_in_a_unit_circle.pdf}
  \caption{Hexagon inscribed in a unit circle}
  \label{fig:hiuc}
\end{figure}

we start off by taking an inscribed regular hexagon inside a unit circle as shown in the Figure \ref{fig:hiuc} and then find the side length of the regular hexagon. \\

Let $d$ be the diameter of the unit circle, $l_{6}$ and $l_{12}$ be the lengths of each side of regular hexagon and dodecagon respectively.
Then, it is evident from [[fig 2]] that\\
$$d = 2 \cdot l_{6}$$
$$2 = 2 \cdot l_{6}$$
$$l_{6} = 1$$
\bigskip

Now, we find the perimeter $p_{6}$ of the hexagon.\\
$$p_{6} = 6 \cdot l_{6}$$
$$p_{6} = 6 \cdot 1$$
$$p_{6} = 6$$
This perimeter $p$ is used as an approximation for the circumference of the unit circle.\\

To find the lower bound of $\pi$, we take the ratio of the perimeter $p_{6}$ and diameter $d$ of the unit circle as follows:\\
$$\pi = \frac{p_{6}}{d}$$
$$\pi = \frac{6}{2}$$
$$\pi = 3$$

% Show the figure 3 containing an inscribed hexagon and dodecagon in a circle.
\begin{figure}[h!]
  \includegraphics[width=\linewidth]{creating_dodecagon_from_hexagon.pdf}
  \caption{Creating a dodecagon from a hexagon inscribed in a circle}
  \label{fig:creating dodecagons}
\end{figure}

\bigskip
\bigskip
\bigskip
\bigskip
\bigskip
To get a better approximation of the lower bound of $\pi$, we need to increase the number of sides of the inscribed polygon. For this, we can double the number of sides of the inscribed hexagon and form a regular dodecagon. \\


We can construct a dodecagon from the hexagon by grabbing the midpoints of the sides of the hexagon and then pulling them to the circumference of the unit circle in the radially outward direction as shown in Figure \ref{fig:creating dodecagons}. \\


To find the length of the sides of the regular dodecagon, we first establish a
general relation between the sides of inscribed n-gon and 2n-gon. Then we use side length
of the hexagon to determine the side length of the dodecagon.

\begin{figure}[h!]
  \includegraphics[width=\linewidth]{sides-relation.pdf}
  \caption{Relation between sides of n-gon and 2n-gon}
  \label{fig:sides relation}
\end{figure}

% Insert the figure 4 here. Figure 4 shows the relation between the side of a n-gon and a 2n-gon.

From Figure \ref{fig:sides relation}, we can clearly see that $\bigtriangleup$AOC and $\bigtriangleup$ABC are right angled triangles. By using pythagorean theorem in $\bigtriangleup$AOC, we get
$$OC^{2}+AC^{2}=OA^{2}$$
$$x^{2}+(\frac{l_{n}}{2})^{2} = 1^{2}$$
$$x = \sqrt{1-\frac{(l_{n})^{2}}{4}}$$

Now using pythagorean theorem in $\bigtriangleup$ABC, we get\\
$$AB^{2}+BC^{2}=AC^{2}$$
$$(\frac{l_{n}}{2})^{2}+(1-x)^{2} = (l_{2n})^{2}$$
$$(\frac{l_{n}}{2})^{2} + 1-2x+x^{2} = (l_{2n})^{2}$$
$$\frac{(l_{n})^{2}}{4}+1-2\cdot\sqrt{1-\frac{(l_{n})^{2}}{4}}+(1-\frac{(l_{n})^{2}}{4})=(l_{2n})^{2}$$
$$2-2\cdot\sqrt{1-\frac{(l_{n})^{2}}{4}}=(l_{2n})^{2}$$
$$2-2\cdot\sqrt{\frac{4-(l_{n})^{2}}{4}}=(l_{2n})^{2}$$
$$2-\sqrt{4-(l_{n})^{2}}=(l_{2n})^{2}$$
$$l_{2n}=\sqrt{2-\sqrt{4-(l_{n})^{2}}}$$
this is the required relationship between each side of a n-gon and a 2n-gon.\\


Now, we already know that $l_{6}=1$. By using this value in the derived relation, we can find the side $l_{12}$ of the regular dodecagon as follows:\\
$$l_{12} = \sqrt{2-\sqrt{4-(1)^{2}}}$$
$$l_{12} = \sqrt{2-\sqrt{3}}$$

Again, we find the perimeter $p_{12}$ of the dodecagon.
$$p_{6}=6\cdot l_{12}$$
$$p_{6}=6\sqrt{2-\sqrt{3}}$$

Now, we can find the better approximation of $\pi$ as follows:
$$\pi = \frac{p_{12}}{d}$$
$$\pi = \frac{6\sqrt{2-\sqrt{3}}}{2}$$
$$\pi = 3\sqrt{2-\sqrt{3}}$$

In this iterative fashion, we can continue to find length of sides of a regular 24-gon, then a 48-gon and so on. And these lengths can be used to find the lower and upper bounds of $\pi$ in each of the n-gons as described above.\\

We can also start with an inscribed and circumscribed square (a 4-gon) and make use of regular octagons (8-gons), then 16-gons, and so on to approximate the lower and upper bounds of $\pi$. Or we can also begin with a regular pentagon, making use of the following relation:
$$\cos(\frac{\pi}{5})=\frac{1+\sqrt{5}}{4}$$
and continue to proceed in the similar way to find the approximations of upper and lower bounds of $\pi$ using inscribed and circumscribed regular decagons (10-gon), 20-gon, and so on.\\

Similarly, we proceed with the circumscribed polygons and find the upper bound of $\pi$.

\subsection{Computer Program}  

We can perform the above iterative process by the means of a
computer program where a user is asked for the number of sides of the starting n-gon as well as the number of iterations. The program determines an
estimate of $\pi$ based on the Archimedes method of approximating $\pi$. The source
code for Archimedes method is represented as follows:

\begin{verbatim}
# Archimedes Method of Approximating Pi using circumscribed and inscribed
regular n-gons inside a circle

# m --> number of iterations
# n0 --> number of sides of the starting regular n-gon
# diameter --> diameter of the unit circle
# ln --> side length of starting inscribed n-gon
# Ln --> side length of starting circumscribed n-gon
# l2n --> side length of inscribed 2n-gon
# L2n --> side length of circumscribed 2n-gon
# num_sides --> number of sides of the n-gon
# cn --> circumference of the inscribed n-gon
# Cn --> circumference of the circumscribed n-gon
# lower_pi --> lower bound of pi
# upper_pi --> upper bound of pi

from math import sqrt

def print_approx_pi(m, n0):
    for i in range(m+1):
        diameter = 2
        ln = 1 if n0 == 6 else (sqrt(10-2*sqrt(5))/2 if n0 == 5 else sqrt(2))
        Ln = 2/sqrt(3) if n0 == 6 else ((2*sqrt(10-2*sqrt(5)))/(sqrt(5)+1)
        if n0 == 5 else 2)
        l2n = determine_l2n(i, ln, n0)
        L2n = determine_L2n(i, Ln, n0)
        num_sides = (2**i)*n0
        cn = num_sides*l2n
        Cn = num_sides*L2n
        lower_pi = cn/diameter
        upper_pi = Cn/diameter
        print("lower bound (iteration ", i, ") = ", lower_pi, end = "")
        print()
        print("upper bound (iteration ", i, ") = ", upper_pi, end ="")
        print("\n\n***************************************\n")
		
def determine_l2n(m, l, n0):

    if m == 0:
        return l if n0 == 6 else (sqrt(10-2*sqrt(5))/2 if n0 == 5 else sqrt(2))

    l2n = sqrt(2-sqrt(4-l**2))
    return determine_l2n(m-1, l2n, n0)
	
def determine_L2n(m, l, n0): 
    if m == 0:
        return 2/sqrt(3) if n0 == 6 else ((2*sqrt(10-2*sqrt(5)))/(sqrt(5)+1)
        if n0 == 5 else 2)
		
    l2n = (2*l)/(2+sqrt(4+l*l))
    return determine_L2n(m-1, l2n, n0)
	
n0 = int(input("Enter the initial value (4, 5 or 6): "))
m = int(input("Enter the number of iterations: "))
print()
print_approx_pi(m, n0)
	
\end{verbatim}
The above program can be understood by the following explanation:

\begin{enumerate}
\item Inputs: user gives the number of sides of the starting regular n-gon and
the number of iterations as an input.
\item Program Main Functions: it consists of three major functions which are
print\_approx\_pi(), determine\_l2n() and determine\_L2n().
\begin{enumerate}
\item determine\_l2n(): it takes number of iterations, side length of starting
inscribed n-gon and number of sides in the starting n-gon as an input. It produces
the side length of inscribed 2n-gon as an output.
\item determine\_L2n(): it takes number of iterations, side length of starting
circumscribed n-gon and number of sides in the starting n-gon as an input. It produces the
side length of circumscribed 2n-gon as an output.
\item print\_approx\_pi(): it mainly calls the determine\_l2n() and determine\_L2n()
to obtain the lengths of each side of inscribed and circumscribed 2n-
gons and then calculates perimeter of the inscribed and circumscribed 2n-gons. It then generates the lower and upper bounds of $\pi$ by taking the ratios of the perimeters of polygons and the diameter of the unit circle.
\end{enumerate}
\item Outputs: lower and upper bounds of $\pi$ are generated.
\end{enumerate}


\section {Results and Analysis}
%we can represent the results in the form of a table, or graph, etc. Graph (approx value vs number of sides) representing the result of each iteration would be a better way to represent the results in this project.

\begin{table}[h!]
  \begin{center}
    \caption{Approximation of $\pi$ using a square as a starting polygon}
    \bigskip
    \label{tab:table1}
    \begin{tabular}{c|c|c|c}
      \textbf{sides} & \textbf{iterations} & \textbf{lower bound} & \textbf{upper bound}\\ % <-- added & and content for each column
      \hline
      4 & 0 & 2.82842712474619 & 3.9999999999999996\\ % <--
      8 & 1 & 3.0614674589207183 & 3.3137084989847603\\ % <--
      16 & 2 & 3.121445152258052 & 3.182597878074528\\ % <--
      32 & 3 & 3.1365484905459393 & 3.151724907429256\\ % <--
      64 & 4 & 3.140331156954753 & 3.1441183852459043\\ % <--
    \end{tabular}
  \end{center}
\end{table}

We see that when a square is taken as a starting polygon, both lower and upper bounds approach $\pi$ as we increase the number of iterations from $0$ to $4$. At each iteration, the number of sides of the n-gon are doubled and thus, increased accuracy of the approximation is achieved. So, we observe that the approximation of $\pi$ increases as the number of sides increase.\\
\bigskip
\bigskip
\bigskip
\bigskip
\bigskip
\bigskip


\begin{table}[h!]
  \begin{center}
    \caption{Approximation of $\pi$ using a regular pentagon as a starting polygon}
    \bigskip
    \label{tab:table2}
    \begin{tabular}{c|c|c|c}
      \textbf{sides} & \textbf{iterations} & \textbf{lower bound} & \textbf{upper bound}\\ % <-- added & and content for each column
      \hline
      5 & 0 & 2.938926261462366 & 3.6327126400268037\\ % <--
      10 & 1 & 3.090169943749474 & 3.249196962329063\\ % <--
      20 & 2 & 3.1286893008046173 & 3.1676888064907254\\ % <--
      40 & 3 & 3.1383638291137976 & 3.1480682729847373\\ % <--
      80 & 4 & 3.140785260725489 & 3.143208560613571\\ % <--
      160 & 5 & 3.141390793700528 & 3.141996443420316\\ % <--
      320 & 6 & 3.1415421878878775 & 3.141693589371422\\ % <--
      640 & 7 & 3.1415800371187137 & 3.1416178868055566\\ % <--
      1280 & 8 & 3.1415894994691733 & 3.1415989618481333\\ % <--
      2560 & 9 & 3.1415918650594596 & 3.141594230651528\\ % <--
      5120 & 10 & 3.141592456457199 & 3.1415930478550487\\ % <--
    \end{tabular}
  \end{center}
\end{table}

We see that when a pentagon is taken as a starting polygon, both lower and upper bounds approach $\pi$ as we increase the number of iterations from $0$ to $10$. At each iteration, the number of sides of the n-gon are doubled and thus, increased accuracy of the approximation is achieved. So, we observe that the approximation of $\pi$ increases as the number of sides increase.

\begin{table}[h!]
  \begin{center}
    \caption{Approximation of $\pi$ using a regular hexagon as a starting polygon}
    \bigskip
    \label{tab:table3}
    \begin{tabular}{{c|c|c|c}}
      \textbf{sides} & \textbf{iterations} & \textbf{lower bound} & \textbf{upper bound}\\ % <-- added & and content for each column
      \hline
      6 & 0 & 3.0 & 3.4641016151377544\\ % <--
      12 & 1 & 3.1058285412302498 & 3.2153903091734723\\ % <--
      24 & 2 & 3.132628613281237 & 3.1596599420975\\ % <--
      48 & 3 & 3.139350203046872 & 3.146086215131435\\ % <--
      96 & 4 & 3.14103195089053 & 3.1427145996453683\\ % <--
      192 & 5 & 3.1414524722853443 & 3.1418730499798233\\ % <--
      384 & 6 & 3.141557607911622 & 3.141662747056848\\ % <--
      768 & 7 & 3.141583892148936 & 3.141610176604689\\ % <--
      1536 & 8 & 3.1415904632367617 & 3.1415970343215256\\ % <--
      3072 & 9 & 3.1415921060430483 & 3.141593748771352\\ % <--
      6144 & 10 & 3.1415925165881546 & 3.141592927385097\\ % <--
      12288 & 11 & 3.1415926186407894 & 3.1415927220386135\\ % <--
      24576 & 12 & 3.1415926453212157 & 3.141592670701998\\ % <--
      49152 & 13 & 3.1415926453212157 & 3.141592657867844\\ % <--
      98304 & 14 & 0 & 3.1415926546593056\\ % <--
      196608 & 15 & 0 & 3.1415926538571712\\ % <--
    \end{tabular}
  \end{center}
\end{table}
\bigskip
\bigskip
We see that when a hexagon is taken as a starting polygon, both lower and upper bounds approach $\pi$ as we increase the number of iterations from $0$ to $13$. At each iteration, the number of sides of the n-gon are doubled and thus, increased accuracy of the approximation is achieved. But beyond iteration $13$, lower bound and the upper bounds are observed to be $0$. This may happen due to the limitation of the data type (floating point numbers) used to store the side length in the computer program. It is possible that after iteration $13$, the interpreter may not be able to differentiate between the extremely small values of the side lengths of the polygons and hence produce $0$ as the upper and lower bounds.

\section {CONCLUSION}
%summarize findings, acknowledge potential future research, indicate any results which could not be obtained.



\section {REFERENCES}
%should not be numbered
Cite the following:\\
--pythagorean theorem\\


\section {APPENDICES}
includes computer code, any other related info which is not strongly related but you want the reader to have access to. Number them in a different notation from sections.

\end{document}
